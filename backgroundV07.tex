\section{Background}\label{ch:bg}
\subsection{Geometry of Three-dimensional Orientations}
\label{subsec:geometry}

Three-dimensional orientation data consist of  observations belonging to the group $SO(3)$ 
of all $3\times 3$ rotation matrices, where an element $\bm R$ in $SO(3)$ is an orthogonal 
$3\times 3$ matrix (i.e., $\bm{R}^\top \bm{R}=\bm{I}_{3\times3}$) with
determinant one.  As $SO(3)$ is a Lie group, its elements live on a differentiable
manifold.  This aspect is helpful in understanding the two different geometric approaches 
for estimating the central location $\bm{S} \in SO(3)$ from a sample of orientation data, referred to here 
as the \textit{intrinsic}  and  the \textit{embedding} estimation approaches (see also \citet{jupp89} and \citet{mardia00} for analogs with directional data).

The rotation group $SO(3)$ is not closed under routine addition or scalar multiplication (i.e., operations natural to statisticians). Hence, statistical estimation approaches often \textit{embed} the rotation group into the higher-dimensional linear space consisting of all $3\times 3$ real matrices, denoted as $\M(3)$.  Doing so enables the use of the familiar Euclidean geometry (and ``averaging'' notions) to define standard distance measures  and loss criteria for obtaining location estimators (see Section~\ref{subsec:metrics} and the estimators given in Sections \ref{subsec:pam} and \ref{subsec:med}).  This embedding technique has been largely  applied by statisticians, typically resulting in the projected arithmetic mean of Section~\ref{subsec:pam}.
See, for example, \cite{downs72, khatri77} and \cite{jupp79, jupp89}; the Bayesian estimator used in \cite{bingham10} is also a concrete example of this
approach.


Alternatively, \textit{intrinsic} estimation approaches use Riemannian geometry to define distances that account for the innate topology or curvature of the space $SO(3)$.  In the intrinsic approach, each rotation from $SO(3)$ is
associated with a skew-symmetric matrix $\bm{\Phi}(\bm{W})$, defined  as
\[
  \bm{\Phi}(\bm{W}) = \left[ \begin{array}{ccc} 0 & -w_3 & -w_2\\
  w_3 & 0 & -w_1\\
  w_2 & w_1 & 0\\
  \end{array}
 \right]
\]
for $\bm{W}=(w_1, w_2, w_3)^\top \in \mathbb{R}^3$. That is, through a so-called exponential operator,
we map  $\bm{\Phi}(\bm{W})$ to a rotation matrix as
\[
  \exp[\bm{\Phi}(\bm{W})] = \bm{I}_{3\times3}\cos(r) + \sin(r) \bm{\Phi}(\bm{U}) + (1-\cos r) \bm{U} \bm{U}^\top
\]
where $r=\|\bm{W}\|$ and $\bm{U} =\bm{W}/\|\bm{W}\| $.  The space $\mathfrak{so}(3)$ of all skew-symmetric matrices forms the tangent space (Lie-algebra) of $SO(3)$,
which is closed under familiar summation and scalar multiplication operations in the usual (i.e., element-wise) manner. The fact that $SO(3)$ is a differentiable manifold allows a distance measure (i.e., the geodesic distance in Section~\ref{subsec:metrics}) to be defined between points in $SO(3)$ according to Riemannian geometry. The resulting geodesic distance underlies the ``geometric" location estimator for $\bm{S}$ commonly found in computer science \citep{fletcher08, fletcher09, hartley11} and engineering applications \citep{manton04}; see Sections~\ref{section:ltwo} and \ref{subsec:lone}.

Before leaving this section, it is helpful to note that each rotation matrix $\bm{R}$ can be uniquely associated with a
pair $(r,\bm{U})$, where $r\in[0,\pi]$ and $\bm{U}\in\mathbb{R}^3$, $\|\bm{U}\|=1$, through
\begin{equation}
\label{eqn:angleaxis}
 \bm{R} = \bm{R}(r,\bm{U}) = \exp[\bm\Phi(\bm{U} r)] \in SO(3).
\end{equation}
This is the so-called axis-angle representation of $\bm{R}$, where $\bm{R}$ is represented by rotating the coordinate axis $\bm{I}_{3 \times 3}$ about the axis $\bm{U}\in\mathbb{R}^3$ by the angle $r$. In the materials science literature,
$\bm{U}$ and $r$ are commonly referred to as the misorientation axis and misorientation angle of $\bm R$ with respect to  $\bm{I}_{3 \times 3}$; see \cite{randle03}.



