 % !TEX root = Stanfill_CoDA.tex
\section{Location Estimators}\label{sec:estimators}
This section describes four estimators for the location parameter $\bm{S}\in SO(3)$ corresponding to orientation data generated by the model in (\ref{eqn:1}). The estimators are based on two different choices. First, the choice whether to use the embedding approach, i.e. to base the estimator on the distance metric (\ref{d_E})  (see Sections~\ref{subsec:pam} and \ref{subsec:med}) or alternatively to use the intrinsic approach employing the distance metric as defined in (\ref{d_R}) (see Sections~\ref{section:ltwo} and \ref{subsec:lone})). The second choice concerns the decision-theoretic loss functions, i.e., either using squared distances (see Sections~\ref{subsec:pam} and \ref{section:ltwo}) or  absolute distances (see Sections~\ref{subsec:med} and \ref{subsec:lone}).  The extent to which the choice of geometry or loss function matters in the estimation of $\bm{S}$ will be an important aspect explored in Section~\ref{ch:simulation}.  A summary of the four estimators and their properties is given in Table~\ref{tab:ests.sum}.




\subsection{The projected arithmetic mean}
\label{subsec:pam}

We begin with the definition of the arithmetic mean for orientation data, as its analog is most frequently encountered in the statistical literature for directional data \citep[e.g. see][]{mardia00}.   For a sample of $n$ random rotations $\bm{R}_i\in SO(3)$, $i=1,2,\dots,n$, this mean-type estimator is defined as
\begin{equation}\label{est:pam}
\ProjMean=\argmin_{\bm{S}\in
SO(3)}\sum_{i=1}^n\Edist^2(\bm{R}_i,\bm{S})=\argmax_{\bm{S}\in
SO(3)}\tr(\bm{S}^{\top}\bar{\bm{R}})
\end{equation}
where $\bar{\bm{R}}=\frac{1}{n}\sum_{i=1}^n\bm{R}_i$. The estimator is obtained by minimizing the sum of the squared distances in the Euclidean sense in the ambient space $\M(3)$, which then is projected back into $SO(3)$. \citet{moakher02}, who studied the mathematical characteristics of this estimator in detail, therefore refers to it as the \textit{projected arithmetic mean}.    This estimator's appeal lies in its simplicity and statistically intuitive nature, though it has been noted that the estimator is not invariant under rigid transformations \citep[see][]{moakher02}.  However, this estimator does correspond to the maximum likelihood estimator of $\bm{S}$ when the symmetrically distributed rotation errors in (\ref{eqn:1}) follow a matrix Fisher distribution \citep{jupp79}.  \citet{leon06} also derived this estimator as the method of moment estimator under a Cayley  distribution, and \citet{bingham09} showed that the projected arithmetic mean corresponds to the maximum quasi-likelihood estimator for orientation data with rotation errors arising from the circular-von Mises-based distribution.  

 \citet{arun87} and \citet{horn88} independently offered algorithms to find this matrix.  \citet{umeyama91} refined their solutions and also considered special cases such as $\det(\bar{\bm R})=0$.

\subsection{The projected median}
\label{subsec:med}

A modification of the estimator from Section~\ref{subsec:pam} is obtained by replacing the squared distances in \eqref{est:pam} with absolute distances, leading to a median-type estimator defined as
\begin{equation}\label{est:med}
\ProjMedian=\argmin_{\bm{S}\in
SO(3)}\sum_{i=1}^n\Edist(\bm{R}_i,\bm{S}).
\end{equation}
We will refer to this estimator of $\bm{S}$ as the \textit{projected median}. \\

Although median-type estimators have been proposed for high dimensional directional data, these estimator were defined for circular and spherical data only and had not been extended to $SO(3)$ data as done in (\ref{est:med}). For spherical data following a von Mises-Fisher distribution \citet{chan93}, for example,  considered  the so-called normalized spatial median of \cite{ducharme87}, an $L_1$ estimator and the Fisher median \red{\cite{Fisher85}} for estimating the central direction of data points on the sphere.   The authors conclude that the normalized spatial median estimator is preferable for spherical data under the von Mises-Fisher model. For additional information on (\ref{est:med}) and related estimators for directional data on the circle or sphere (e.g. the mediancentre \citep{gower74} or the Weber point \citep{bajaj88}),  see \citet{durocher09}. 

We further propose an algorithm to compute the projected median.  We base our method on the Weiszfeld algorithm originally given by \cite{weiszfeld37}.  The algorithm requires an initial value that does not equal any sample point. For the purpose of speeding up computing time we use $\ProjMean$ as the starting point. Note that the solution is generally not sensitive to the choice of starting points unless the data exhibit extreme spread.
\begin{enumerate}
\item Set $\widehat{\bm S}=\ProjMean$ and choose an arbitrarily small stopping rule $\varepsilon$.
\item For $i=1,\ldots,n$ compute $\bm s_i=\bm R_i-\widehat{\bm S}$.
\item Calculate
\[
\bar{\bm R}_W=\frac{\sum_{i=1}^n\bm R_i/||\bm s_i||_F}{\sum_{i=1}^n1/||\bm s_i||_F}
\]
which we call the weighted mean with respect to $\widehat{\bm S}$.
\item Define $\widehat{\bm S}_{\text{new}}$ to be the $\mathcal{M}(3)$ projection of $\bar{\bm R}_W$.
\item If $\varepsilon>||\widehat{\bm S}-\widehat{\bm S}_{\text{new}}||_F$ return $\ProjMedian=\widehat{\bm S}_{\text{new}}$; otherwise set $\widehat{\bm S}=\widehat{\bm S}_{\text{new}}$ and return to step 2.
\end{enumerate}

\subsection{The geometric mean}
\label{section:ltwo}
As sketched in Section~\ref{subsec:geometry}, the Lie group property of $SO(3)$ provides us with a convenient transform from $SO(3)$
into the tangent space $\mathfrak{so}(3)$ that is closed under
addition and scalar multiplication.  Obtaining the median or mean
in this transformed space and projecting the result back to $SO(3)$ corresponds to the rotation that minimizes the first and second order Riemannian
distances, respectively \citep{karcher77, moakher02, fletcher08, fletcher09}.  \citet{karcher77} made use of Riemannian manifolds to compute what is often called the Riemannian
center of mass.  \citet{moakher02} applied Karcher's ideas to
rotation matrices and defined
\begin{equation}\label{est:ltwo}
\GeomMean=\argmin_{\bm{S}\in
SO(3)}\sum_{i=1}^n\Rdist^2(\bm{R}_i,\bm{S}).
\end{equation}
which was termed as the \textit{geometric mean}.  Note that the solution to  \eqref{est:ltwo} may not be
unique. Uniqueness is tied to the property of geodesic convexity of the objective function in \eqref{est:ltwo}. For more information, we refer to \citet{moakher02}.  Additionally, \eqref{est:ltwo} generally does not have a closed-form solution making this estimator much more computationally intensive than its Euclidean counterpart (the projected arithmetic mean of Section~\ref{subsec:pam}).  We used the algorithm proposed by \citet{manton04} in our simulation study.

\subsection{The geometric median}
\label{subsec:lone}
The median-type counterpart to the geometric mean was defined first in the context of
spherical data by \citet{fisher85} as the point on the sphere that minimizes the sum of the arc lengths to all
observations in the sample.   For this type of data, the resulting estimator is known as the spherical median,
 which is a special case of the generalized median in $\R^d$
proposed by \citet{gower74}.   For spherical data, an alternative formulation to the
spherical median has been given by \citet{liu92} in the framework of
data depth leading, however, to the same solution.

We give an adaptation of the spherical median to rotation matrices. 
Recall that the shortest geodesic path between two rotations ${\bm R_1}$, ${\bm R_2}$ is given by the Riemannian distance $\Rdist(\bm R_1,\bm R_2)$.  Thus the rotation matrix analog of the \cite{fisher85} spherical
median can be defined as
\begin{equation}\label{est:lone}
\GeomMedian=\argmin_{\bm{S}\in
SO(3)}\sum_{i=1}^n\Rdist(\bm{R}_i,\bm{S});
\end{equation}
see also \cite{fletcher08, fletcher09}.  We refer to this estimator of $\bm{S}$ as the \textit{geometric median.}  \citet{hartley11} offers an algorithm to find the geometric median in $SO(3)$.


\begin{table}[h]
\caption{A summary of the estimators presented and their properties.  \label{tab:ests.sum}}
\begin{center}
\begin{tabular}{ lclclcl}\hline
\rule[2mm]{0mm}{1mm} \textbf{name} & & \textbf{symbol} & & \textbf{distance metric} &&\textbf{cost function}\\ \hline \hline 
\rule[2mm]{0mm}{6mm} Projected Arithmetic Mean & & $\ProjMean$ & & Euclidean &&$\sum_{i=1}^n\Edist^2$  \\
\rule[2mm]{0mm}{6mm} Projected Median & & $\ProjMedian$ & & Euclidean && $\sum_{i=1}^n\Edist$ \\
\rule[2mm]{0mm}{6mm} Geometric Mean & & $\GeomMean$&  & Riemannian && $\sum_{i=1}^n\Rdist^2$\\0
\rule[2mm]{0mm}{6mm} Geometric Median & & $\GeomMedian$&  & Riemannian &&$\sum_{i=1}^n\Rdist$ \\[-7mm] 
\rule[2mm]{0mm}{6mm} & & & & \\ \hline
\end{tabular}
\end{center}
\end{table}
