% !TEX root = Stanfill_CoDA.tex
\section{Recommendations and Conclusions}\label{sec:disc}

The scientific literature suggests a variety of approaches to estimate the central orientation $\bm S$ given a random sample of three-dimensional orientations from \eqref{eqn:1}. These approaches differ largely with respect to the geometry (Riemannian vs.~Euclidean) of estimation, assumptions about the underlying data-generating mechanism, and the choice of loss function when defining suitable estimators. The main goal of this paper was to explore the extent to which these choices affect the estimation of $\bm S$. 
Our simulation study showed that  the underlying data-generating mechanism guides our choice of loss function.  For the circular-von Mises-based model median-type estimators perform better while for the Cayley and matrix Fisher model the mean-type estimators show less estimation error and variability. As noted in Section~\ref{ch:intro} the applied sciences generally pursue estimation of $\bm S$ without considering the distributional underpinnings. Restricting ourselves to the three rotation distributions under consideration,  if indeed nothing is known about the underlying data-generating mechanism we suggest to use either median-type estimator, where the proposed median, the Euclidean based estimator $\GeomMedian$, emerges as a good overall choice. Its overall estimation error, even under mis-specification, will be much less than the potential estimation error resulting from either mean-type estimator. The extent to which all four estimators disagree depends on the variability in the rotation data; the estimators differ more when the circular variance $\nu$ is large and tend to agree more as the data become more concentrated.  
   
Further studies could be extended to include location estimation in non-symmetric distributional models for rotations as we considered common models for symmetric perturbations around $\bm S$ in \eqref{eqn:1}. Another important consideration is the extension of the studied point estimators to building confidence regions for the location parameter $\bm S$. One possibility would be resampling-based confidence regions, but this requires theoretical development for the estimator's sampling distributions and improvements in computing time before this can be practically implemented. \\

%=======
%% !TEX root = Stanfill_CoDA.tex
%\section{Recommendations and Conclusions}\label{sec:disc}
%
%The scientific literature suggests a variety of approaches to estimate the central orientation $\bm S$ given a random sample of three-dimensional orientations from \eqref{eqn:1}. These approaches differ largely with respect to the geometry (Riemannian vs.~Euclidean) in which the estimation is done, assumptions about the underlying data-generating mechanism and the choice of loss function when defining suitable estimators. The main goal of this paper was to explore the extent to which these choices affect the estimation of $\bm S$. 
%Our simulation study showed that  the underlying data-generating mechanism guides our choice of loss function.  For the circular-von Mises-based model median-type estimators perform better while for the Cayley and matrix Fisher model the mean-type estimators show less estimation error and variability. As noted in Section~\ref{ch:intro} the applied sciences generally pursue estimation of $\bm S$ without considering the distributional underpinnings. This can be a pitfall. Restricting ourselves to the three rotation distributions under consideration,  if indeed nothing is known about the underlying data-generating mechanism we suggest to use either median-type estimator, i.e. $\ProjMedian$ or $\GeomMedian$. The overall estimation error, even under mis-specification will be much less than the potential estimation error resulting from either mean-type estimator. The effect the geometry has on the precision of the estimates also depends on the underlying distributional model, however, to a smaller degree. %The Riemannian distance metric  $\Rdist$ is generally not recommended with a mean-type estimator due to the resulting increase in variability in the estimation errors. This fact holds across all three distributions  and is much expressed than for any of the other three estimators.
%In summary, the extent to which all four estimators disagree relative to each other depends on the circular variance $\nu$; the estimators differ more when $\nu$ is large and tend to agree more as the data become more concentrated.  
%   
%The scope of this study can be extended to other frequently encountered distributional models, especially because we restricted the simulation to symmetric perturbations around $\bm S$ in \eqref{eqn:1}. At least as important, if not even more, is the extension of the studied point estimators to interval estimators. The latter requires a significant improvement in computing time before it can be practically implemented. \\
%
%>>>>>>> 2ae524a3512a18ca58f9808d9a1c66a11687f0c3
\noindent \textbf{Acknowledgements:} The authors wish to thank Melissa Bingham and the Ames Laboratory for collecting and providing the EBSD data. We also gratefully acknowledge the suggestions made by the reviewers, the Associate Editor and Editor, all of which significantly improved the manuscript.