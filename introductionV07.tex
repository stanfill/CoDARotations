 % !TEX root = Stanfill_CoDA.tex
\section{Introduction}\label{ch:intro}

Data in the form of $3 \times 3$ rotation matrices find application in several scientific areas, 
such as biomedical engineering, computer visioning, and geological and materials sciences, where such data represent the positions of objects within some three-dimensional
reference frame.  For example, \citet{rancourt00} examine rotation matrix data in studying body 
positions whilst operating machinery. \cite{fletcher09} consider this type of  orientation data in 
magnetic resonance imaging and in shape analysis; similar examples  can be found in \cite{schwartz05}, \cite{pierrynowski09},  \cite{dai10},  or \cite{hadani11}.  The data in our illustrative 
example to follow arise from a study in materials science, where $3 \times 3$ rotations represent the orientations of cubic crystals on the micro-surface of a metal specimen as measured through electron backscatter diffraction (EBSD) and ``grains" within metals are composed of crystals which roughly share a common orientation; see \cite{randle03} for details on EBSD data.  
     
From a sample of orientations, an important interest is often the estimation of a main or central orientation $\bm S$.  That is, letting the rotation group $SO(3)$ denote the collection of all $3\times 3$ rotation matrices, observations $\bm{R}_1,\ldots,\bm{R}_n \in SO(3)$ can be conceptualized as a random sample from a \textit{location model} \vspace{-0.25cm}
\begin{equation}
\label{eqn:1}
\mathbf{R}_i = \bm{S} \bm{E}_i, \quad i=1,\ldots,n,
\end{equation}\vspace{-1cm}\\
where $\bm S \in SO(3)$ is the {\it fixed} parameter of interest indicating an orientation of central tendency, and $\bm{E}_1,\ldots,\bm{E}_n \in SO(3)$ denote i.i.d.~{\it random} rotations which symmetrically perturb $\bm{S}$. The data-generating model in \eqref{eqn:1} is a rotation-matrix analog of a location model for scalar data $Y_i = \mu + e_i$, where $\mu \in \mathbb{R}$ denotes a mean and $e_i \in \mathbb{R}$ denotes an additive error symmetrically distributed around zero.  This representation \eqref{eqn:1} for orientations is quite common and, in fact,
a variety of parametric models exist for describing symmetrically distributed rotations $\bm{E}_i$, such as the symmetric matrix Fisher distribution \citep{downs72}, the symmetric Cayley distribution
\citep{leon06}  and the circular-von Mises-based rotation distribution \citep{bingham09}   in the statistics literature, as well as Bunge's Gaussian distribution \citep{bunge82}, the isotropic Gaussian
distribution \citep{matthies88, savyolova95} and the de la Vall\'{e}e Poussin distribution \citep{Schaeben97} in the materials science literature. Our goal in this paper is to summarize and compare the most frequently proposed approaches for the point estimation of $\bm S$ based on a sample of orientation data generated by \eqref{eqn:1}.  Depending on the scientific literature, the approaches can be quite different.

The topic of location estimation has received considerable attention for directional data on circles or spheres \citep[see][]{fisher53, karcher77, khatri77, fisher85, ducharme87, bajaj88, liu92, chan93, mardia00}, but less is known about  estimator properties with rotation data.
As a compounding factor, several current approaches to estimating $\bm S$ have arisen out of literatures having differing statistical and geometrical emphases.  In the  applied sciences literature, estimators of $\bm S$ are typically based on
{\it non-Euclidean} (i.e., Riemannian) geometry, such as the \emph{geometric mean} \citep{arun87, horn88, umeyama91, moakher02} or, more recently, the \emph{geometric median} \citep{fletcher08, fletcher09}.
Preferences may depend on potential outliers in the data, but such suggestions for estimating $\bm S$
often do {\it not} consider the potential impact of the
underlying data-generating mechanism.   On the other hand,
approaches in the statistics literature tend to motivate an estimator for $\bm S$  through likelihood or moment-estimation principles applied to a specifically assumed distributional model (e.g., matrix Fisher or Cayley distribution) for the symmetric rotation errors $\bm E_i$ \citep{downs72, jupp79, leon06, bingham10}. Almost always, this estimator turns out to be 
a \emph{projected arithmetic mean} based on {\it Euclidean} geometry. Hence, in addition to possible distributional assumptions, more fundamental divisions in estimation approaches may be attributable to different geometrical perspectives with rotation data.  

Considering the potential effects of an underlying data generation model as well as the choice of geometry (i.e., Euclidean vs.~Riemannian), the above discussion indicates a need to investigate and identify good point estimators for rotation data.  In particular, because estimators in the applied sciences literatures  are often selected without decision-theoretical considerations based on underlying distributions, it is of interest to understand how different location estimators behave across common distributions for rotations.  In this paper, we evaluate four estimators for $\bm S$ in the context of the location model \eqref{eqn:1}. These are either mean- or median-type estimators and based either on Euclidean or Riemannian geometry; the Euclidean-based median estimator is introduced for the first time for $SO(3)$ data. Its inclusion is natural and its performance can be generally quite good so that this estimator may be broadly recommendable (as will be demonstrated).   Through simulation, we compare how these estimators perform with respect to three common probability models for symmetric rotation errors as defined in \eqref{eqn:1}, namely the circular-von Mises-based distribution, the symmetric matrix Fisher distribution and the symmetric Cayley distribution.  The matrix Fisher is arguably the most common distribution in the statistics literature \citep[see][]{chikuse03}. While not noted previously, the symmetric Cayley and the de la Vall\'{e}e Poussin distribution are in fact the same; the de la Vall\'{e}e Poussin distribution has been advocated in the materials science literature \citep{Schaeben97}.   The circular-von Mises-based distribution is included because the distribution is non-regular and has been applied to EBSD data \citep{bingham09}.  We describe how properties of error distributions for rotation data, in particular their variability and tail behavior, translate into performance differences among point estimators of $\bm S$.\\
The remainder of the manuscript is organized  as follows.  Section~\ref{ch:bg} provides a brief background on the geometry of rotations and different distance metrics that can be used to assess overall estimation bias.   Section~\ref{sec:estimators} then describes the location estimators for rotation data  and compares their geometric underpinnings, which  serves to unify some of the existing estimation literature.  Section~4 explains the design of the simulation study followed by a summary of our main findings in Section~5. Section~6 provides an illustration of the estimation methods for EBSD data in a materials science application. We provide concluding remarks and future research possibilities in Section~7. Accompanying supplemental materials are available online.
