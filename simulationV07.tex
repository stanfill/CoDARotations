 % !TEX root = Stanfill_CoDA.tex
\section{Simulation Study}\label{ch:simulation}

Section~\ref{subsec:simdesign} gives an outline the simulation design.  Section~\ref{subsec:genRR} briefly describes the parametric distributional models used in the study for describing symmetric rotation errors (cf.~(\ref{eqn:1})) with differing variability. 

 %and Section~\ref{subsec:algorithms} provides the algorithms used for numerically evaluating the location estimators of Section~\ref{sec:estimators}.

\subsection{Design of Simulation Study}
\label{subsec:simdesign}
To compare the performance of the proposed location estimators for determining the central direction $\bm{S}$ given a sample of size $n$, we generated random rotation error samples  $\bm E_1, \ldots, \bm E_n$  in model \eqref{eqn:1} with sizes $n=10, 50, 100$ and 300. In \eqref{eqn:1}, without loss of generality, we set the location parameter $\bm S=\bm I_{3\times 3}$ (the identity matrix) and considered circular variances(denoted $\nu$) of $\nu=0.25$, $0.50$ and $0.75$. 

%\red{that is  a bit abrupt -- should we move all of the above discussion relating to the densities to the next subsection? HH}
For each combination of sample size, spread $\nu$ and choice of distribution, we generated 1,000 samples and for each sample estimated the central direction  $\bm S=\bm I_{3\times 3}$ using the four estimators on each sample.  The next section continues with an introduction to the distributions under consideration.


\subsection{Generating Random Rotations in the Location Model}
\label{subsec:genRR}
%As mentioned in the introduction, 
We wish to compare estimators of the (fixed) location parameter $\bm{S}\in SO(3)$ under three common distributional models for describing symmetric rotation errors $\bm{E}\in SO(3)$ in a data model  $\bm{R}=\bm{S}\bm{E}$ (cf.~(\ref{eqn:1})): the symmetric matrix Fisher \citep{langevin05, downs72, khatri77, jupp79}, the symmetric Cayley  \citep{Schaeben97, leon06} and the circular-von Mises-based distribution \citep{bingham09}. A general construction approach exists for random rotations that are symmetrically distributed around the identity matrix $\bm{I}_{3 \times 3}$; see \cite{watson83, bingham09} and \cite{hielscher10}.  To this end, let $\bm{U}\in\mathbb{R}^3$ represent a point chosen uniformly on the unit sphere and, independently, generate a random angle $r$ according to some circular density $C(r|\kappa)$ on $(-\pi,\pi]$, which is symmetric around 0 and where $\kappa$ denotes a concentration parameter governing the spread of the circular distribution.  Then, define a random rotation as $\bm{E}=\bm{E}(\bm{U},r)$ using the constructive definition (\ref{eqn:angleaxis}) (i.e., $\bm{E}$ represents the position of $\bm{I}_{3\times 3}$ upon rotating the standard coordinate frame in $\mathbb{R}^3$ about the random axis $\bm{U}$ by the random angle $r$). The resulting rotation $\bm{E}$ will be symmetrically distributed and its distributional type (i.e., matrix Fisher, Cayley or circular-von Mises-based) is determined by the form of the circular density $C(r|\kappa)$ for the (misorientation) angle $r$.  

\begin{center} 
\begin{table}[h]
\caption{Circular densities with respect to the Lebesgue measure and circular variance $\nu$.  \label{tab:ang.dens}}
\small{
\begin{tabular}{ lclcl}\hline
\textbf{Name} & & \textbf{Density} $C(r |\kappa)$ & & \textbf{Circular variance $\nu$}\\ \hline \hline 
\rule[2mm]{0mm}{6mm} Cayley & & $\frac{1}
{\sqrt{\pi}} \frac{\Gamma(\kappa+2)}{\Gamma(\kappa+1/2)} 
2^{-(\kappa+1)} (1+\cos r)^\kappa(1-\cos r)$ & & $\frac{3}
{\kappa+2}$ \\
\rule[2mm]{0mm}{6mm} matrix Fisher & & $\frac{1}{2\pi[\mathrm{I_0}(2\kappa)-\mathrm{I_1}(2\kappa)]}e^{2\kappa 
\cos(r)}[1-\cos(r)]$ & & 
$\frac{3\mathrm{I}_0(2\kappa)-4\mathrm{I}_1(2\kappa)+\mathrm{I}_2(2\kappa)}
{2[\mathrm{I}_0(2\kappa)-\mathrm{I}_1(2\kappa)]}$ \\
\rule[2mm]{0mm}{6mm} circular-von Mises & & $\frac{1}{2\pi \mathrm{I_0}(\kappa)}e^{\kappa\cos(r)}$&  & 
$\frac{\mathrm{I_0}(\kappa)-\mathrm{I_1}(\kappa)}{\mathrm{I_0}(\kappa)}$ \\[-7mm] 
\rule[2mm]{0mm}{6mm} & & & & \\ \hline
\end{tabular}}
\end{table}
\end{center}

The circular desnities are given in Table~\ref{tab:ang.dens}, where $\mathrm{I_p}(\cdot)$ denotes the Bessel function of order $p$ defined as  $\mathrm{I_p}(\kappa)=\frac{1}{2\pi}\int_{-\pi}^{\pi}\cos(pr)e^{\kappa\cos r}dr$. To ease the interpretation of the simulation results in Section~\ref{sec:results},  we choose not to present results in terms of $\kappa$ but instead use the circular variance defined as $\nu=1-\rho$ as a measure of spread, where $\rho=\mathrm{E}[\cos(r)]$, commonly referred to as the mean resultant length. This allows us to compare the performance of the estimators for densities exhibiting the same spread.  The values of $\kappa$ corresponding to the chosen circular variances are given in Table~\ref{tab:kappas}.  \\


\begin{table}[h!]
\begin{center}
\caption{Values of $\kappa$ for each rotational distribution corresponding to the circular variances.  \label{tab:kappas}}\vspace{-0.4cm}
\begin{tabular}{l l ccc}\hline
{\bf Distribution} & & \multicolumn{3}{c}{\bf Circular variance} \\
& & $\nu=0.25$ &$\nu=0.50$ & $\nu=0.75$\\ \hline \hline
Cayley & & 10.00 & 4.00 & 2.00 \\
matrix Fisher & & 3.17 & 1.71 & 1.15\\
circular-von Mises & & 2.40 & 1.16 & 0.52\\ \hline
\end{tabular}
\end{center}
\end{table}

The density, with respect to the Haar measure, for each distribution of a random rotation given a circular variance of $0.75$ is plotted in Figure~\ref{fig:Haar}.  The Haar measure (or uniform distribution on $SO(3)$) acts as the dominating measure for rotations and the symmetric nature of the random rotation $\bm E_i=\bm E_i(\bm U,r)$ means that its density $f(\bm E_i|\nu)=f(r|\nu)$ can be plotted in terms of the misorientation angle $r$ of $\bm E_i$ in \eqref{eqn:angleaxis}, which is common in materials science \citep{matthies88, savyolova95}.  Density plots for the other circular variances we consider in our simulation study are similar and therefore omitted.

%Old figures the AE didn't like.  Wanted us to show whole c-vM distribution and add zoom plot here to save space.
% \begin{figure}[h!]
% \centering
% \subfloat[$\nu=0.25$]{\label{fig:cayden}\includegraphics[width=0.33\textwidth]{Var25DensityHaar.pdf}}
% \subfloat[$\nu=0.50$]{\label{fig:fishden}\includegraphics[width=0.33\textwidth]{Var5DensityHaar.pdf}}
% \subfloat[$\nu=0.75$]{\label{fig:vonmden}\includegraphics[width=0.33\textwidth]{Var75DensityHaar.pdf}}
% \caption{Density functions for the three rotational distributions with respect to the Haar measure. The solid line corresponds to the density of matrix Fisher distribution, the dashed line the density of the circular-von Mises-based distribution and the dotted line to the density of the Cayley distribution.}
% \label{fig:Haar}
% \end{figure}

\begin{figure}[h!]
\centering
\subfloat[Overview]{\label{fig:zoomout}\includegraphics[width=0.33\textwidth]{Var75DensityHaarFull.pdf}}
\subfloat[Mode Behavior]{\label{fig:body}\includegraphics[width=0.33\textwidth]{Var75DensityBox.pdf}}
\subfloat[Tail Behavior]{\label{fig:denzoom}\includegraphics[width=0.33\textwidth]{Var75DensityZoomNoGuide.pdf}}
\caption{Comparison of the distributions of interest for $\nu=0.75$.  Clearly the circular-von Mises based-distribution has the highest peak \red{isn't that even a pole in this parameterization?}, and surprisingly it has the heaviest tail as well.}
\label{fig:Haar}
\end{figure}

In Figure \ref{fig:zoomout} we see that the circular-von Mises-based distribution has a much higher peak \red{HH I believe that's a pole; we divide by the Haar measure and in zero that gives us a pole} than the other two distributions.  Figure \ref{fig:body} demonstrates that the Cayley distribution has the lowest peak, and there appear to be two points where the three distributions cross one another.  Finally, Figure \ref{fig:denzoom} illustrates the tail behavior of the distributions and indicates the the circular-von Mises distribution has the heaviest tail of the three distributions. 

\blue{Figure \ref{eyeballs} shows sphere plots for 100 rotations each sampled from one of the three angular distributions we discuss. From left to right, samples are shown from a Cayley distribution, the matrix Fisher distributions, and the circular-von Mises distribution. Each of these distributions was adjusted to have an angular variance of $\nu = 0.25$. 
Because rotation matrices are orthogonal, each of their columns (and rows) represents elements on the unit sphere. Here, we show only the first column of each of the rotation matrices. But since we are assuming independence of angle and main direction, any of the other directions is expected to show a similar pattern. The pattern that we see between the distributions re-enforces previous findings. While the circular-von Mises distributions has the highest mass close to the main direction in the center of the circles, it also shows the largest numbers of `outlying' rotations, i.e. rotations with a large radial distance to the center. The sphere plots for the Cayley and the matrix-Fisher distribution are very similar -- from the previous discussion we know that the Cayley distribution has a lower concentration of mass in the center than the matrix-Fisher distribution -- this is, with a bit of goodwill, also visible from the point pattern on the sphere plots. 

\begin{figure}[htbp]
\includegraphics[width=.3\linewidth]{eye-cayley}
\includegraphics[width=.3\linewidth]{eye-fisher}
\includegraphics[width=.3\linewidth]{eye-vmises}
\caption{\label{eyeballs}Sphere plots of the first column for each of the three angular distributions under an angular variance of  $\nu = 0.25$}
\end{figure}
}


In the simulations to follow, for generating random rotation errors based on the construction above, we used different samplers to randomly generate angles $r\in(-\pi,\pi]$ from a given circular density, recalling that the form of $C(r|\kappa)$ depends on the intended symmetric distribution for the rotation errors $\bm{E}$.  We defer these details to Section~\ref{sec:appendix1} of the Appendix.

%\subsection{Estimator Algorithms}
%\label{subsec:algorithms}

%Two of the estimators of interest require a projection from the space of all $3\times 3$ matrices, $\mathcal{M}(3)$, into $SO(3)$.  We perform this projection according to the following method.  For a matrix $\bm M\in\mathcal{M}(3)$ let $\lambda_1 \geq \lambda_2 \geq \lambda_3$ denote the ordered eigenvalues of $\bm M^\top\bm M$ with corresponding eigenvectors $ \bm u_1, \bm u_2, \bm u_3$.  Defining $\bm U=[\bm u_1, \bm u_2, \bm u_3]$ and provided that $\det(\bm M)\neq 0$, the unique projection of $\bm M$ into $SO(3)$ is given by
%\[
%\bm M\bm{U} \text{diag}\left(\frac{1}{\sqrt{\lambda_1}},\frac{1}{\sqrt{\lambda_2}},\frac{\text{sign}\left[\det(\bm M)\right]}{\sqrt{\lambda_3}}\right) \bm{U}^\top.
%\]
%We will refer to this as the $\mathcal{M}(3)$ projection algorithm. To calculate $\widehat{\bm{S}}_P$ given a sample $\{\bm{R}_1,\dots,\bm{R}_n\}\subset SO(3)$ we compute the $\mathcal{M}(3)$ projection of $\bar{\bm R}=\frac{1}{n}\sum_{i=1}^n\bm{R}_i$.  See \citet{arun87,horn88} and \citet{umeyama91} for an introduction and refinements of this solution including special cases such as $\det(\bar{\bm R})=0$.

%To compute the projected median $\widetilde{\bm S}_P$ we use a Weiszfeld-type algorithm originally given by \cite{weiszfeld37}.  The algorithm requires an initial value that does not equal any sample point. For the purpose of speeding up computing time we use $\widehat{\bm S}_P$ as the starting point. Note that the solution is generally not sensitive to the choice of starting points unless the data exhibit extreme spread.
%\begin{enumerate}
%\item Set $\widehat{\bm S}=\widehat{\bm S}_{P}$ and choose an arbitrarily small stopping rule $\varepsilon$.
%\item For $i=1,\ldots,n$ compute $\bm s_i=\bm R_i-\widehat{\bm S}$.
%\item Calculate
%\[
%\bar{\bm R}_W=\frac{\sum_{i=1}^n\bm R_i/||\bm s_i||_F}{\sum_{i=1}^n1/||\bm s_i||_F}
%\]
%which we call the weighted mean with respect to $\widehat{\bm S}$.
%\item Define $\widehat{\bm S}_{\text{new}}$ to be the $\mathcal{M}(3)$ projection of $\bar{\bm R}_W$.
%\item If $\varepsilon>||\widehat{\bm S}-\widehat{\bm S}_{\text{new}}||_F$ return $\widetilde{\bm{S}}_P=\widehat{\bm S}_{\text{new}}$; otherwise set $\widehat{\bm S}=\widehat{\bm S}_{\text{new}}$ and return to step 2.
%\end{enumerate}

%The second set of algorithms is based on the relationship between $SO(3)$ and its tangent space $\mathfrak{so}(3)$ through the principal logarithm of a rotation $\bm R$, see Sections~\ref{subsec:geometry} and \ref{subsec:metrics}. The rotation that minimizes the sum of the squared geodesic distances, $\widehat{\bm S}_G$, for a given tolerance level $\varepsilon$ can be found as follows \citep[see][]{manton04}.
%\begin{enumerate}
%\item Initiate a value for $\widehat{\bm S}$, e.g. $\widehat{\bm S}= \widehat{\bm S}_P$ .
%\item Calculate $\bm s=\frac{1}{n}\sum_{i=1}^n\Log(\widehat{\bm{S}}^\top\bm R_i)$.
%\item If $||\bm s||_F<\varepsilon$ return $\widehat{\bm S}_{G}=\widehat{\bm S}$; otherwise set $\widehat{\bm S}=\widehat{\bm S}\exp(\bm s)$ (see Section~\ref{subsec:geometry}), repeat step 2 and re-evaluate.
%\end{enumerate}


%Finally, to find the minimizer of the first order geodesic distances we use an algorithm developed by \citet{hartley11}.  Similar to $\widetilde{\bm S}_{P}$, this algorithm requires a starting point not in the sample, so we use $\widehat{\bm S}_{G}$ to save computational time.
%\begin{enumerate}
%\item Set $\widehat{\bm S}=\widehat{\bm S}_{G}$.
%\item For each sample point compute $\bm s_i=\Log(\widehat{\bm S}^\top\bm R_i)$.
%\item Set 
%\[
%\bm\delta=\frac{\sum_{i=1}^n \bm s_i/||\bm s_i||_F}{\sum_{i=1}^n 1/||\bm s_i||_F}.
%\]
%\item If $||\bm\delta||_F<\varepsilon$ then return $\widetilde{\bm S}_{G}=\widehat{\bm S}$; otherwise update $\widehat{\bm S}=\exp(\bm\delta)\widehat{\bm S}$ and return to step 2.
%\end{enumerate}

